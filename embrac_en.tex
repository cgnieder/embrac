% arara: pdflatex
% arara: makeindex: { sort: true, style: embrac_en.ist }
% arara: biber
% arara: pdflatex
% arara: pdflatex
% --------------------------------------------------------------------------
% the EMBRAC package
% 
%   Upright Brackets in Emphasized Text
% 
% 2012/06/29
% --------------------------------------------------------------------------
% Clemens Niederberger
% Web:    https://bitbucket.org/cgnieder/acro/
% E-Mail: contact@mychemistry.eu
% --------------------------------------------------------------------------
% Copyright 2011-2012 Clemens Niederberger
% 
% This work may be distributed and/or modified under the
% conditions of the LaTeX Project Public License, either version 1.3
% of this license or (at your option) any later version.
% The latest version of this license is in
%   http://www.latex-project.org/lppl.txt
% and version 1.3 or later is part of all distributions of LaTeX
% version 2005/12/01 or later.
% 
% This work has the LPPL maintenance status `maintained'.
% 
% The Current Maintainer of this work is Clemens Niederberger.
% --------------------------------------------------------------------------
% The acro package consists of the files
%  - acro.sty, acro_en.tex, acro_en.pdf, README
% --------------------------------------------------------------------------
% If you have any ideas, questions, suggestions or bugs to report, please
% feel free to contact me.
% --------------------------------------------------------------------------
%
% if you want to compile this documentation you'll need the document class
% `cnpkgdoc' which you can get here:
%    https://bitbucket.org/cgnieder/cnpkgdoc/
% the class is licensed LPPL 1.3 or later
%
\documentclass[DIV10,toc=index,toc=bib]{cnpkgdoc}
\docsetup{
  pkg      = [biblatex]embrac,
  code-box = {
    backgroundcolor  = gray!7!white ,
    skipbelow        = .6\baselineskip plus .5ex minus .5ex ,
    skipabove        = .6\baselineskip plus .5ex minus .5ex ,
    roundcorner      = 3pt ,
  } ,
  gobble   = 1
}
\usepackage[osf]{libertine}
\usepackage{fnpct}
\cnpkgcolors{
  main   => cnpkgred ,
  key    => yellow!40!brown ,
  module => cnpkgblue ,
  link   => black!90
}

\renewcommand*\othersectionlevelsformat[3]{%
  \textcolor{main}{#3\autodot}\enskip}
\renewcommand*\partformat{%
  \textcolor{main}{\partname~\thepart\autodot}}


\ExplSyntaxOn
\NewDocumentCommand \Default {g}
  {
    \hfill\llap
      {
        \IfNoValueTF { #1 }
          {(initially~empty)}
          {Default:~\code{#1}}
      }
    \newline
  }
\ExplSyntaxOff

\addcmds{
  AddEmph,
  DeleteEmph,
  EmbracOff,
  EmbracOn,
  printbibliography,
  RenewEmph
}

\usepackage[backend=biber,style=alphabetic]{biblatex}
\addbibresource{\jobname.bib}
\usepackage{filecontents}
\begin{filecontents}{\jobname.bib}
@book{bringhurst04,
  title     = {The Elements of Typographic Style},
  author    = {Robert Bringhurst},
  year      = {2004},
  version   = {3.2},
  isbn      = {978-0-88179-205-5},
  publisher = {Hartley \&\ Marks, Canada}
}
@article{dtk12-dw,
  author  = {Dominik Waßenhoven},
  title   = {Aufrechte Klammern in kursivem Text},
  journal = {Die \TeX{}nische Komödie},
  volume  = {2},
  year    = {2012},
  pages   = {51--53}
}
@online{lefloch11,
  author  = {Bruno Le Floch},
  title   = {Upright parentheses in italic text},
  url     = {http://tex.stackexchange.com/a/13057/5049},
  date    = {2011-03-09},
  urldate = {2012-06-28}
}
\end{filecontents}

% index:
\usepackage{makeidx}
\begin{filecontents}{\jobname.ist}
 heading_prefix "{\\bfseries "
 heading_suffix "\\hfil}\\nopagebreak\n"
 headings_flag  1
 delim_0 "\\dotfill "
 delim_1 "\\dotfill "
 delim_2 "\\dotfill "
 delim_r "\\textendash"
 suffix_2p "\\nohyperpage{\\,f.}"
 suffix_3p "\\nohyperpage{\\,ff.}"
\end{filecontents}
\makeindex

\usepackage{fixfoot}
\DeclareFixedFootnote\manythanks{Many thanks for his kind permission to use it!}

\RenewEmph{[}[][.02em]{]}[.055em][-.08em]
\RenewEmph{(}[-.01em][.04em]{)}[.04em][-.05em]
\begin{document}

\section{License and Requirements}\secidx{License}\secidx{Requirements}
\embrac is placed under the terms of the LaTeX Project Public License,
version 1.3 or later (\url{http://www.latex-project.org/lppl.txt}).
It has the status ``maintained.''

\embrac loads and needs the packages \paket{expl3}, \paket{xparse} and
\paket{l3keys2e}, the first is part of the \paket{l3kernel} and the others are
part of the \paket{l3packages}.
\secidx*{License}\secidx*{Requirements}

\section{About}\label{sec:about}\secidx{About}
\begin{zitat}[{\cite[85]{bringhurst04}}]
 Parentheses and brackets are not letters, and it makes little sense to speak of
 them as roman or italic. There are vertical parentheses and sloped ones, and
 the parentheses on italic fonts are almost always sloped, but vertical parentheses
 are generally to be preferred. That means they must come from the roman font,
 and may need extra spacing when used with italic letterforms.
 
 The sloped square brackets usually found on italic fonts are, if anything, even
 less useful than sloped parentheses. If, perish the thought, there were a book
 or film entitled \emph*{The View from My [sic] Bed}, sloped brackets might be
 useful as a way of indicating that the brackets and their contents are actually
 part of the title. Otherwise, vertical brackets should be used, no matter whether
 the text is roman or italic: \textquote{The View from My \emph{[sic]} Bed} and
 \textquote{\emph{the view from my [sic] bed}.\kern.05em}
\end{zitat}

Both this quote from \citetitle{bringhurst04} by Robert Bringhurst and the
article \citetitle{dtk12-dw} by Dominik Waßenhoven \cite{dtk12-dw} inspired
this package. Indeed, \embrac heavily borrows from the code \citeauthor{dtk12-dw}
provided in his article\manythanks. The code was originally provided by Bruno
Le~Floch\manythanks\ as an answer to a question on \{TeX.SE\} \cite{lefloch11}.

\embrac tries to redefine \cmd{emph} and \cmd{textit} in a way that neither
parentheses nor square brackets are sloped. 

\begin{beispiel}
 \emph{This is emphasized [sic] text.} \\
 \emph{This is emphasized text (as you can see).}
\end{beispiel}
\secidx*{About}

\section{How it Works}\secidx{How it Works}
Both the commands \cmd{emph} and \cmd{textit} are redefined:
\begin{beschreibung}
 \Befehl{emph}*{<emphasized text>}
 \Befehl{textit}*{<text in italics>}
\end{beschreibung}
They now both have a \code{*} argument that restores the original behaviour.
Otherwise they're used just as before.

Let's see the example again:
\begin{beispiel}
 \emph{This is emphasized [sic] text.} \\
 \emph{This is emphasized text (as you can see).}
\end{beispiel}
As you can see you don't have to do anything apart from loading \embrac in your
preamble. Well -- that's actually not entirely true. You have to take care of the
kerning of the parentheses and brackets. Otherwise things could look worse with
\embrac than without.

The following example demonstrates one point why you have to be very careful
when using \embrac. Certain parenthesis-letter combinations might need adjustments
of the kerning:
\begin{beispiel}
 \emph{This is (just) emphasized text.} \\
 \emph{This is (\kern.1em just) emphasized text.}
\end{beispiel}
This of course strongly depends on the font you've chosen.
\secidx*{How it Works}

\section{Adding More Brackets \&\ Adjusting the Kerning}
\secidx{Adding More Brackets}\secidx{Adjusting the Kerning}
If you want you can change the behaviour of \embrac. Maybe it should only affect
squared brackets? Or curly braces, too? This can be done with the following
commands. They all work locally which means if used inside a group outside of it
everything stays what it was.
\begin{beschreibung}
 \Befehl{AddEmph}{<o>}\oa{<io-kern>}\oa{<oo-kern>}\ma{<c>}\oa{<ic-kern>}\oa{<oc-kern>}
 \Befehl{RenewEmph}{<o>}\oa{<io-kern>}\oa{<oo-kern>}\ma{<c>}\oa{<ic-kern>}\oa{<oc-kern>}
 \Befehl{DeleteEmph}{<o>}\ma{<c>}
\end{beschreibung}
Here \code{<o>} means \emph{opening bracket} and \code{<c>} means \emph{closing
bracket}. The optional \code{<io-kern>} is inserted after the opening bracket
(inner opening), \code{<oo-kern>} is inserted before it (outer opening).
\code{<ic-kern>} is inserted before the closing bracket (inner closing),
\code{<oc-kern>} after it (outer closing).

\embrac initially makes these definitions:
\begin{beispiel}[code only]
 % add some defaults:
 \AddEmph{[}{]}[.04em][-.12em]
 \AddEmph{(}[-.04em]{)}[][-.15em]
\end{beispiel}
Please notice that \cmd{RenewEmph} only replaces the specified kerning values
and leaves all others as they've been specified before.

For example, this document uses ``Linux Libertine O''%
\footnote{\url{http://www.linuxlibertine.org/}} both for the roman and the italic
font and redefines them in this way, still not sure these are the best values:
\begin{beispiel}[code only]
 \RenewEmph{[}[][.02em]{]}[.055em][-.08em]
 \RenewEmph{(}[-.01em][.04em]{)}[.04em][-.05em]
\end{beispiel}

You can change them as you wish, but be careful with the kerning! What's too less
for one letter might be too much for others:
\begin{beispiel}
 \RenewEmph{(}[.1em]{)}[.1em]
 \emph{This is (just) emphasized text.} \\ % looks ok
 \emph{This is emphasized text (as you can see).} % looks bad
\end{beispiel}

Deleting a pair removes it completely from \embrac's mechanism. All information
about the pair and it's kerning values will be lost. So if you want the effects
to be temporary use grouping.
\begin{beispiel}
 \DeleteEmph[]
 \emph{This is emphasized [sic] text.} \\
 \emph{This is emphasized text (as you can see).}
\end{beispiel}
\secidx*{Adding More Brackets}\secidx*{Adjusting the Kerning}

\section{biblatex Compatibility}\secidx{biblatex Compatibility}
This is \embrac's only package option:
\begin{beschreibung}
 \Option{biblatex}{\default{true}|false}\Default{false} If set to true parentheses
 and squared brackets as defined by \paket{biblatex} with \cmd{bibleftparen},
 \cmd{bibrightparen}, \cmd{bibleftbracket} and \cmd{bibrightbracket} are treated
 the same way (if not removed from the treatment with \cmd{DeleteEmph}).
\end{beschreibung}

\begin{beispiel}
 % this document uses \usepackage[biblatex]{embrac}
 \emph{Let's cite the quote from section \ref{sec:about} again: \cite{bringhurst04}}.
\end{beispiel}
\secidx*{biblatex Compatibility}

\section{Turn \embrac Temporarily Off}\secidx{Turn \embrac Temporarily Off}
Redefining \cmd{emph} and \cmd{textit} is not without danger. Sometimes you might
find that you'd want to keep the original definition for a small portion of your
document. You can use the following commands whose functions are obvious, I guess:
\begin{beschreibung}
 \Befehl{EmbracOff}
 \Befehl{EmbracOn}
\end{beschreibung}
Both commands are local.

This document for example uses the following lines because \paket{biblatex}
complained when it tried to typeset the entry for \cite{bringhurst04}. Using
\cmd{DeleteEmph}[] would have worked as well.
\begin{beispiel}[code only]
 \EmbracOff
 \printbibliography
\end{beispiel}
\secidx*{Turn \embrac Temporarily Off}

\section{Watch out!}\secidx{Watch Out}
Please be aware that \embrac does not affect \cmd{itshape} nor the commands
\cmd{em} or \cmd{it} (which you shouldn't use in a \LaTeX\ document, anyway).
\begin{beispiel}
 \itshape This is italic [sic] text.
\end{beispiel}
\secidx*{Watch Out}

\EmbracOff
\printbibliography


\setindexpreamble{Section titles are indicated \textbf{bold}, packages
\textsf{sans serif}, commands \code{\textbackslash\textcolor{code}{brown}}
 and options \textcolor{key}{\code{yellow}}.\par\bigskip}
\printindex
\end{document}